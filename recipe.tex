% Recipe: Processing framework overview
\documentclass[11pt]{article}

% --- Basic setup ---
\usepackage[T1]{fontenc}
\usepackage[utf8]{inputenc}
\usepackage{lmodern}
\usepackage{geometry}
\geometry{margin=1in}
\usepackage{microtype}
\usepackage{hyperref}
\usepackage{enumitem}
\usepackage{booktabs}
\usepackage{longtable}
\usepackage{graphicx}
\usepackage{xcolor}

% --- Title info ---
\title{Processing Recipe: PressureProcess Framework}
\author{SAPPHiRe Project}
\date{\today}

\begin{document}
\maketitle

\section*{Purpose}
This recipe describes how to use the PressureProcess framework to clean, calibrate, and analyse wall-pressure measurements. It is written to be a concise, repeatable guide for running the processing pipeline and understanding its inputs, outputs, and assumptions.

\section*{Scope}
The steps below cover: (1) raw data ingestion, (2) cleaning and quality control, (3) calibration, (4) feature extraction, and (5) output packaging for downstream analysis and reporting.

\section*{Inputs}
\begin{itemize}[leftmargin=*,nosep]
  \item Raw pressure time series (binary, .mat file) with metadata (sampling rate, sensor IDs, test conditions).
  \item Calibration references (from anechoic).
\end{itemize}

\section*{Processing Overview}
\begin{enumerate}[leftmargin=*,itemsep=0.2em]
  \item \textbf{Ingest:} Load raw files and verify metadata consistency.
  \item \textbf{Clean:} De-trend, remove invalid segments, handle dropouts/outliers, and log QC flags.
  \item \textbf{Calibrate:} Apply sensor calibrations and tonal adjustments per the calibration plan.
  \item \textbf{Transform:} Windowing and spectral transforms as required (e.g., PSD, band levels).
  \item \textbf{Extract:} Compute features and metrics needed for reports and models.
  \item \textbf{Package:} Write standardized outputs with provenance and processing parameters.
\end{enumerate}

\section*{Key Parameters}
\begin{longtable}{@{}p{0.35\linewidth}p{0.6\linewidth}@{}}
\toprule
\textbf{Parameter} & \textbf{Description} \\
\midrule
Sampling rate & Expected sampling frequency for each run. \\
Window type/length & Window function and length for spectral calculations. \\
Overlap & Window overlap fraction for FFT-based estimates. \\
Filter settings & Bandpass/notch filters used in cleaning. \\
Calibration factors & Per-sensor gain/phase or tonal adjustments. \\
QC thresholds & Rules for outliers, dropouts, or clipping detection. \\
\bottomrule
\end{longtable}

\section*{Outputs}
\begin{itemize}[leftmargin=*,nosep]
  \item Cleaned time series and QC logs.
  \item Calibrated signals and spectral products (PSD, bands, SPL).
  \item Feature tables for analysis and reporting.
  \item Figures (optional): spectra, time histories, and QC summaries.
\end{itemize}

\section*{Reproducibility}
Record all inputs, parameter values, and code versions used. Outputs should include a provenance block with: run ID, input hashes, calibration version, and the processing date.

\section*{How to Run}
\begin{enumerate}[leftmargin=*,itemsep=0.2em]
  \item Prepare a manifest of runs and verify raw data availability.
  \item Configure processing parameters (sampling rate, filters, calibration).
  \item Execute the cleaning and calibration steps.
  \item Generate spectra/features and review QC plots.
  \item Export outputs and archive provenance logs.
\end{enumerate}

\section*{Notes}
Adjust the steps to match specific test campaigns. If tonal calibration is required, follow the tonal calibration plan and ensure the correct reference files are used.

Injest:
\begin{itemize}
    \item Set up the file structure and paths for raw data and outputs
    \item 
\end{itemize}

\end{document}
