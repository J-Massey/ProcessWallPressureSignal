\documentclass[aspectratio=169,9pt]{beamer}
% Beamer layout
\setbeamersize{text margin left=0.4cm, text margin right=0.2cm}

% Math & fonts
\usepackage{amsmath,amssymb}
\usepackage{mathpazo}   % Palatino text + math

% Tables & units
\usepackage{booktabs}
\usepackage{siunitx}

% TikZ (one load, all libs)
\usepackage{tikz}
\usetikzlibrary{arrows.meta,calc,positioning,shapes.geometric,shapes.misc}

% Algorithms
\usepackage{algorithm}
\usepackage[noend]{algpseudocode}
\usepackage{float}

% Media
\usepackage{multimedia}
\usepackage{animate}

% Bibliography
\usepackage{natbib}


\newcommand{\shortdate}{\the\month-\the\day}
\graphicspath{{../../figures/.}}

\include{setup}
\definecolor{cardinalred}{RGB}{140,21,21}
%--- Custom footline
\setbeamertemplate{footline}{%
\begin{beamercolorbox}[wd=\paperwidth,ht=4ex,dp=2.5ex]{}
    \centering
    \makebox[0.32\paperwidth][l]{\scriptsize\texttt{Spec. \shortdate}}
    \makebox[0.32\paperwidth][c]{\scriptsize\texttt{$^{\dag}$masseyj@stanford.edu}}
    \makebox[0.32\paperwidth][r]{\scriptsize\insertframenumber/\inserttotalframenumber}
  \end{beamercolorbox}
}
\usepackage{algorithm}
\usepackage[noend]{algpseudocode}
\usetikzlibrary{arrows.meta,shapes.geometric,shapes.misc,positioning}

\tikzset{
  >={Stealth},
  proc/.style = {rectangle, rounded corners, draw, align=left, minimum height=8mm, text width=35mm, align=center},
  meas/.style    = {trapezium, trapezium left angle=70, trapezium right angle=110, draw, align=left, minimum height=8mm, text width=19mm, align=center},
  decision/.style = {diamond, aspect=2.2, draw, align=center, inner xsep=1.2ex, inner ysep=1ex, text width=2cm},
  smldec/.style = {diamond, aspect=2.2, draw, align=center, inner xsep=1.2ex, inner ysep=1ex, text width=19mm},
  terminator/.style = {ellipse, draw, align=center, minimum height=8mm, minimum width=16mm},
  line/.style  = {->, line width=0.6pt}
}


%--- Title info
\title{SU update: \\ We have two-point measurements with two streamwise spacings}
\author{JMO Massey$^{\dag}$, F Cabrera-Booman, JC Klewicki, T Jaroslawski, BJ McKeon}
\institute{Center for Turbulence Research \\ Stanford University}
% \thanks{This work was supported by DARPA under the CHAOS program}
\date{\today}

\begin{document}

%--- Title page
\begin{frame}
    \setcounter{framenumber}{0}
    \titlepage
    \vfill
    {\scriptsize \centering Thanks to DARPA for funding this work.\par}
\end{frame}

\begin{frame}

    \begin{itemize}
    %     \item Fix $U_e=14[\mathrm{m/s}]$
        \item $\delta\approx 0.035[\mathrm{m}]$, $U_e \approx14[\mathrm{m/s}]$, $T^+\equiv T u_\tau^2/\nu=10$
    \end{itemize}
    \begin{table}[]
        \centering
        \begin{tabular}{lccccc}
        \toprule
        Pressure (psi) & 0  &  50 & 100 \\
        \midrule
        $u_\tau$[m/s] & 0.58 & 0.47 & 0.52 \\
        $\nu/u_\tau$ [m] & 27$\times 10^{-6}$ & 7.5$\times 10^{-6}$ & 3.7$\times 10^{-6}$ \\
        $\nu$ [m$^2$/s] & 15.7$\times 10^{-6}$ & 3.52$\times 10^{-6}$ & 1.92$\times 10^{-6}$ \\
        $Re_\tau$ & 1,300 & 4,700 & 9,500 \\
        \textcolor{blue}{$f(T^+=10)$ [Hz]} & \textcolor{blue}{2,100} & \textcolor{blue}{4,700} & \textcolor{blue}{14,100} \\
        \bottomrule
        \end{tabular}
    \end{table}

\end{frame}

\begin{frame}
  \frametitle{Pinhole diameters}
  \begin{figure}
    \centering
    \includegraphics[width=0.7\textwidth]{pinholes.pdf}
  \end{figure}

  \begin{itemize}
    \centering
        \item Testing pinhole diameters of $d=$ 2300, 700, 400 $\mu$m
        \item Corresponds to $d^+ \approx$ 85, 93, 108
        \item Under the frozen turbulence assumption, these sit around $T^+\sim 10$
    \end{itemize}
\end{frame}

\begin{frame}
  \frametitle{Pinhole spacings}

  \begin{columns}
    \begin{column}{0.3\textwidth}
      \begin{figure}
        \centering
        \includegraphics[width=\textwidth]{pinhole_spacing.pdf}
      \end{figure}
    \end{column}
    \begin{column}{0.5\textwidth}
      \begin{itemize}
        \item We have two-point measurements at two streamwise spacings: $3.2\delta$ and $2.8\delta$
        \item Herein, we refer to these as `far' and `close' spacings
        \item The spectra are plotted in voltage and haven't yet been converted to pressure
      \end{itemize}
    \end{column}
  \end{columns}
\end{frame}

\section{Raw Data}
\begin{frame}
  \centering
  \vfill
  {\Huge\bfseries \textcolor{cardinalred}{Raw Data}}
  \vfill
\end{frame}


\begin{frame}
  \frametitle{Raw Data [Pa]: $Re_\tau\approx$ 1,300 ($d=$700 $\mu$m)}
  \begin{figure}
    \centering
    \includegraphics[width=0.7\textwidth]{tf_corrected_spectra/700_0psi_raw_spec_far.png}
    \includegraphics[width=0.7\textwidth]{tf_corrected_spectra/700_0psi_raw_spec_close.png}
  \end{figure}
\end{frame}

\begin{frame}
  \frametitle{Raw Data [Pa]: $Re_\tau \approx$ 4,700 ($d=$700 $\mu$m)}
  \begin{figure}
    \centering
    \includegraphics[width=0.7\textwidth]{tf_corrected_spectra/700_50psi_raw_spec_far.png}
    \includegraphics[width=0.7\textwidth]{tf_corrected_spectra/700_50psi_raw_spec_close.png}
  \end{figure}
\end{frame}

\begin{frame}
  \frametitle{Raw Data [Pa]: $Re_\tau \approx$ 9,500 ($d=$700 $\mu$m)}
  \begin{figure}
    \centering
    \includegraphics[width=0.7\textwidth]{tf_corrected_spectra/700_100psi_raw_spec_far.png}
    \includegraphics[width=0.7\textwidth]{tf_corrected_spectra/700_100psi_raw_spec_close.png}
  \end{figure}
\end{frame}


\section{Transfer Functions}
\begin{frame}
  \centering
  \vfill
  {\Huge\bfseries \textcolor{cardinalred}{Transfer Functions}}
  \vfill
\end{frame}

\begin{frame}
  \frametitle{Transfer Function: $Re_\tau\approx$ 1,300 ($d=$700 $\mu$m)}
  \begin{figure}
    \centering
    \includegraphics[width=0.9\textwidth]{tf_calib/700_0psi_H_2cal.png}
  \end{figure}
\end{frame}

\begin{frame}
  \frametitle{Transfer Function: $Re_\tau \approx$ 4,700 ($d=$700 $\mu$m)}
  \begin{figure}
    \centering
    \includegraphics[width=0.9\textwidth]{tf_calib/700_50psi_H_2cal.png}
  \end{figure}
\end{frame}

\begin{frame}
  \frametitle{Transfer Function: $Re_\tau \approx$ 9,500 ($d=$700 $\mu$m)}
  \begin{figure}
    \centering
    \includegraphics[width=0.9\textwidth]{tf_calib/700_100psi_H_2cal.png}
  \end{figure}
\end{frame}

\begin{frame}
  \frametitle{Under development: Combining transfer functions}
  \begin{itemize}
    \centering
    \item The transfer functions appear to be similar, but not identical.
    \item Small discrepancies can lead to large differences in the final spectra
    \item We are working on methods to combine the 4 transfer functions into one robust estimate
    \item With high confidence, we can fuse the two anechoic transfer functions into one robust estimate
  \end{itemize}
\end{frame}

\section{Fused TFs}

\begin{frame}
  \centering
  \vfill
  {\Huge\bfseries \textcolor{cardinalred}{Fused Transfer Functions}}
  \vfill
\end{frame}

\begin{frame}
  \frametitle{Anechoic Fused Transfer Function: $Re_\tau \approx$ 1,300 ($d=$700 $\mu$m)}
  \begin{figure}
    \centering
    \includegraphics[width=0.7\textwidth]{tf_calib/700_0psi_H_anechoic_fused.png}
    \includegraphics[width=0.7\textwidth]{tf_calib/700_0psi_gamma_fuse.png}
  \end{figure}
\end{frame}

\begin{frame}
  \frametitle{Anechoic Fused Transfer Function: $Re_\tau \approx$ 4,700 ($d=$700 $\mu$m)}
  \begin{figure}
    \centering
    \includegraphics[width=0.7\textwidth]{tf_calib/700_50psi_H_anechoic_fused.png}
    \includegraphics[width=0.7\textwidth]{tf_calib/700_50psi_gamma_fuse.png}
  \end{figure}
\end{frame}

\begin{frame}
  \frametitle{Anechoic Fused Transfer Function: $Re_\tau \approx$ 9,500 ($d=$700 $\mu$m)}
  \begin{figure}
    \centering
    \includegraphics[width=0.7\textwidth]{tf_calib/700_100psi_H_anechoic_fused.png}
    \includegraphics[width=0.7\textwidth]{tf_calib/700_100psi_gamma_fuse.png}
  \end{figure}
\end{frame}

\begin{frame}
  \frametitle{Anechoic + In-Situ Fused Transfer Function: $Re_\tau \approx$ 1,300 ($d=$700 $\mu$m)}
  \begin{figure}
    \centering
    \includegraphics[width=0.7\textwidth]{tf_calib/700_0psi_H_fuse_situ.png}
    % \includegraphics[width=0.7\textwidth]{tf_calib/700_atm_gamma_fuse.png}
  \end{figure}
\end{frame}

\begin{frame}
  \frametitle{Anechoic + In-Situ Fused Transfer Function: $Re_\tau \approx$ 4,700 ($d=$700 $\mu$m)}
  \begin{figure}
    \centering
    \includegraphics[width=0.7\textwidth]{tf_calib/700_50psi_H_fuse_situ.png}
    % \includegraphics[width=0.7\textwidth]{tf_calib/700_50psi_gamma_fuse.png}
  \end{figure}
\end{frame}

\begin{frame}
  \frametitle{Anechoic + In-Situ Fused Transfer Function: $Re_\tau \approx$ 9,500 ($d=$700 $\mu$m)}
  \begin{figure}
    \centering
    \includegraphics[width=0.7\textwidth]{tf_calib/700_100psi_H_fuse_situ.png}
    % \includegraphics[width=0.7\textwidth]{tf_calib/700_100psi_gamma_fuse.png}
  \end{figure}
\end{frame}

\section{Resulting Spectra Close Spaced}

\begin{frame}
  \centering
  \vfill
  {\Huge\bfseries \textcolor{cardinalred}{Resulting Spectra}}
  \vfill
\end{frame}

\begin{frame}
  \frametitle{Calibrated Spectra: $Re_\tau \approx$ 1,300 ($d=$700 $\mu$m)}
  \begin{figure}
    \centering
    \includegraphics[width=0.7\textwidth]{tf_corrected_spectra/700_0psi_tf_fused_insitu_spec_close.png}
  \end{figure}
\end{frame}

\begin{frame}
  \frametitle{Calibrated Spectra: $Re_\tau \approx$ 4,700 ($d=$700 $\mu$m)}
  \begin{figure}
    \centering
    \includegraphics[width=0.7\textwidth]{tf_corrected_spectra/700_50psi_tf_fused_insitu_spec_close.png}
  \end{figure}
\end{frame}

\begin{frame}
  \frametitle{Calibrated Spectra: $Re_\tau \approx$ 9,500 ($d=$700 $\mu$m)}
  \begin{figure}
    \centering
    \includegraphics[width=0.7\textwidth]{tf_corrected_spectra/700_100psi_tf_fused_insitu_spec_close.png}
  \end{figure}
\end{frame}

\section{White noise problem ?}

\begin{frame}
  \frametitle{Calibrated Spectra: $Re_\tau \approx$ 1,300 ($d=$700 $\mu$m)}
  \begin{figure}
    \centering
    \includegraphics[width=0.7\textwidth]{tf_calib/700_atm_calib_spec_a2.png}
  \end{figure}
\end{frame}


\begin{frame}
  \frametitle{Calibrated Spectra: $Re_\tau \approx$ 4,700 ($d=$700 $\mu$m)}
  \begin{figure}
    \centering
    \includegraphics[width=0.7\textwidth]{tf_calib/700_50psi_calib_spec_a2.png}
  \end{figure}
\end{frame}

\begin{frame}
  \frametitle{Calibrated Spectra: $Re_\tau \approx$ 9,500 ($d=$700 $\mu$m)}
  \begin{figure}
    \centering
    \includegraphics[width=0.7\textwidth]{tf_calib/700_100psi_calib_spec_a2.png}
  \end{figure}
\end{frame}

\end{document}
