\documentclass[aspectratio=169,9pt]{beamer}
% Beamer layout
\setbeamersize{text margin left=0.4cm, text margin right=0.2cm}

% Math & fonts
\usepackage{amsmath,amssymb}
\usepackage{mathpazo}   % Palatino text + math

% Tables & units
\usepackage{booktabs}
\usepackage{siunitx}

% TikZ (one load, all libs)
\usepackage{tikz}
\usetikzlibrary{arrows.meta,calc,positioning,shapes.geometric,shapes.misc}

% Algorithms
\usepackage{algorithm}
\usepackage[noend]{algpseudocode}
\usepackage{float}

% Media
\usepackage{multimedia}
\usepackage{animate}

% Bibliography
\usepackage{natbib}


\newcommand{\shortdate}{\the\month-\the\day}
\graphicspath{{../../figures/.}}

\include{setup}
\definecolor{cardinalred}{RGB}{140,21,21}
%--- Custom footline
\setbeamertemplate{footline}{%
\begin{beamercolorbox}[wd=\paperwidth,ht=4ex,dp=2.5ex]{}
    \centering
    \makebox[0.32\paperwidth][l]{\scriptsize\texttt{Spec. \shortdate}}
    \makebox[0.32\paperwidth][c]{\scriptsize\texttt{$^{\dag}$masseyj@stanford.edu}}
    \makebox[0.32\paperwidth][r]{\scriptsize\insertframenumber/\inserttotalframenumber}
  \end{beamercolorbox}
}
\usepackage{algorithm}
\usepackage[noend]{algpseudocode}
\usetikzlibrary{arrows.meta,shapes.geometric,shapes.misc,positioning}

\tikzset{
  >={Stealth},
  proc/.style = {rectangle, rounded corners, draw, align=left, minimum height=8mm, text width=35mm, align=center},
  meas/.style    = {trapezium, trapezium left angle=70, trapezium right angle=110, draw, align=left, minimum height=8mm, text width=19mm, align=center},
  decision/.style = {diamond, aspect=2.2, draw, align=center, inner xsep=1.2ex, inner ysep=1ex, text width=2cm},
  smldec/.style = {diamond, aspect=2.2, draw, align=center, inner xsep=1.2ex, inner ysep=1ex, text width=19mm},
  terminator/.style = {ellipse, draw, align=center, minimum height=8mm, minimum width=16mm},
  line/.style  = {->, line width=0.6pt}
}


%--- Title info
\title{SU update: \\ Best estimate}
\author{JMO Massey$^{\dag}$, F Cabrera-Booman, JC Klewicki, T Jaroslawski, BJ McKeon}
\institute{Center for Turbulence Research \\ Stanford University}
% \thanks{This work was supported by DARPA under the CHAOS program}
\date{\today}

\begin{document}

%--- Title page
% \begin{frame}
%     \setcounter{framenumber}{0}
%     \titlepage
%     \vfill
%     {\scriptsize \centering Thanks to DARPA for funding this work.\par}
% \end{frame}

% \begin{frame}

%     \begin{itemize}
%     %     \item Fix $U_e=14[\mathrm{m/s}]$
%         \item $\delta\approx 0.035[\mathrm{m}]$, $U_e \approx14[\mathrm{m/s}]$
%     \end{itemize}
%     \begin{table}[]
%         \centering
%         \begin{tabular}{lccccc}
%         \toprule
%         Pressure (psig) & 0  &  50 & 100 \\
%         \midrule
%         $u_\tau$[m/s] & 0.537 & 0.522 & 0.506 \\
%         $\nu/u_\tau$ [m] & 28$\times 10^{-6}$ & 6.6$\times 10^{-6}$ & 3.8$\times 10^{-6}$ \\
%         $\nu$ [m$^2$/s] & 14.9$\times 10^{-6}$ & 3.42$\times 10^{-6}$ & 1.93$\times 10^{-6}$ \\
%         $Re_\tau$ & 1 263 & 5 340 & 9 178 \\
%         \textcolor{blue}{ROI: $f$ [Hz]} & \textcolor{blue}{100--1 000} & \textcolor{blue}{100--1 000} & \textcolor{blue}{100--1 000} \\
%         \textcolor{blue}{ROI: $T^+$} & \textcolor{blue}{200--20} & \textcolor{blue}{800--80} & \textcolor{blue}{1 300--130} \\
%         \bottomrule
%         \end{tabular}
%     \end{table}

% \end{frame}

% \begin{frame}
%   \frametitle{Pinhole diameters}
%   \begin{figure}
%     \centering
%     \includegraphics[width=0.7\textwidth]{pinholes.pdf}
%   \end{figure}

%   \begin{itemize}
%     \centering
%         \item Testing pinhole diameters of $d=$ 2300, 700, 400 $\mu$m
%         \item Corresponds to $d^+ \approx$ 85, 93, 108
%         \item Under the frozen turbulence assumption, these sit around $T^+\sim 10$
%     \end{itemize}
% \end{frame}

\begin{frame}
  \frametitle{TF eq.}
  \centering
  \begin{align}
    X(t;p_{\textrm{static}}) = \mathcal{F}\{f;p_{PH}\} \quad&\quad Y(t;p_{\textrm{static}}) = \mathcal{F}\{f;p_{NC}\}\\
    H(f;p_{\textrm{static}}) &= XY^*/(YY^*) \\
    \end{align}
\end{frame}

% \begin{frame}
%   \frametitle{Pinhole spacings}

%   \begin{columns}
%     \begin{column}{0.3\textwidth}
%       \begin{figure}
%         \centering
%         \includegraphics[width=\textwidth]{../presentation/pinholes.pdf}
%       \end{figure}
%     \end{column}
%     \begin{column}{0.5\textwidth}
%       \begin{itemize}
%         \item We have two-point measurements at two streamwise spacings: $3.2\delta$ and $2.8\delta$
%         \item Herein, we refer to these as `far' and `close' spacings
%       \end{itemize}
%     \end{column}
%   \end{columns}
% \end{frame}


% \begin{frame}
%   \frametitle{FS noise rejected}
%   \begin{figure}
%       \centering
%       \includegraphics[width=0.9\textwidth]{final/spectra_comparison.png}
%   \end{figure}
% \end{frame}

% \begin{frame}
%   \frametitle{Measurements look great, but the final result is very sensitive to the calibration}
%   \begin{figure}
%       \centering
%       \includegraphics[width=0.7\textwidth]{tf_two_ways/no_flow_close_0psig.png}
%   \end{figure}
% \end{frame}

% \begin{frame}
%   \frametitle{We use an LEM}
%   \begin{equation}
%       \centering
%       |H_{\textrm{cal}}| = |H_{\textrm{wn}}| \times S
%   \end{equation}
%   \begin{equation}
%       \centering
%       20\log S = c_0 +a 20\log \frac{f}{f_{\textrm{ref}}} + b 20\log \frac{\rho}{\rho_{\textrm{ref}}}
%   \end{equation}
%   \begin{itemize}
%     \centering
%     \item Mass scaling and compliance term
%     \item Fit parameters $a$, $b$, $c_0$, and $f_{\textrm{ref}}$ to calibration data at each pressure
%     \item c0 and b do the work
%   \end{itemize}
% \end{frame}

% \begin{frame}
%   \frametitle{Final spectra in ROI}
%   \begin{figure}
%       \centering
%       \includegraphics[width=0.85\textwidth]{final/spectra_comparison_tf_freq_scaled.png}
%   \end{figure}
% \end{frame}

% \begin{frame}
%   \frametitle{Final spectra in ROI}
%   \begin{figure}
%       \centering
%       \includegraphics[width=0.85\textwidth]{final/spectra_comparison_tf_Tplus_scaled.png}
%   \end{figure}
% \end{frame}

% \begin{frame}
%   \frametitle{Speeds}
%   \begin{figure}
%       \centering
%       \includegraphics[width=0.85\textwidth]{pressure_speed/speed_vs_Tplus_raw_absG12_0.png}
%   \end{figure}
% \end{frame}
% \begin{frame}
%   \frametitle{Speeds}
%   \begin{figure}
%       \centering
%       \includegraphics[width=0.85\textwidth]{pressure_speed/speed_vs_Tplus_raw_absG12_50.png}
%   \end{figure}
% \end{frame}

% \begin{frame}
%   \frametitle{Speeds}
%   \begin{figure}
%       \centering
%       \includegraphics[width=0.85\textwidth]{pressure_speed/speed_vs_Tplus_raw_absG12_100.png}
%   \end{figure}
% \end{frame}

% \begin{frame}
%   \frametitle{Shear}
%   \begin{figure}
%       \centering
%       \includegraphics[width=0.85\textwidth]{tau_w/spectra.png}
%   \end{figure}
% \end{frame}






\end{document}
