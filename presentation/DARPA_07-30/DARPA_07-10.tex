\documentclass[aspectratio=169,9pt]{beamer}
% Beamer layout
\setbeamersize{text margin left=0.4cm, text margin right=0.2cm}

% Math & fonts
\usepackage{amsmath,amssymb}
\usepackage{mathpazo}   % Palatino text + math

% Tables & units
\usepackage{booktabs}
\usepackage{siunitx}

% TikZ (one load, all libs)
\usepackage{tikz}
\usetikzlibrary{arrows.meta,calc,positioning,shapes.geometric,shapes.misc}

% Algorithms
\usepackage{algorithm}
\usepackage[noend]{algpseudocode}
\usepackage{float}

% Media
\usepackage{multimedia}
\usepackage{animate}

% Bibliography
\usepackage{natbib}


\newcommand{\shortdate}{\the\month-\the\day}
\graphicspath{{../../figures/.}}

\include{setup}
\definecolor{cardinalred}{RGB}{140,21,21}
%--- Custom footline
\setbeamertemplate{footline}{%
\begin{beamercolorbox}[wd=\paperwidth,ht=4ex,dp=2.5ex]{}
    \centering
    \makebox[0.32\paperwidth][l]{\scriptsize\texttt{Pipe. \shortdate}}
    \makebox[0.32\paperwidth][c]{\scriptsize\texttt{$^{\dag}$masseyj@stanford.edu}}
    \makebox[0.32\paperwidth][r]{\scriptsize\insertframenumber/\inserttotalframenumber}
  \end{beamercolorbox}
}
\usepackage{algorithm}
\usepackage[noend]{algpseudocode}
\usetikzlibrary{arrows.meta,shapes.geometric,shapes.misc,positioning}

\tikzset{
  >={Stealth},
  proc/.style = {rectangle, rounded corners, draw, align=left, minimum height=8mm, text width=35mm, align=center},
  meas/.style    = {trapezium, trapezium left angle=70, trapezium right angle=110, draw, align=left, minimum height=8mm, text width=19mm, align=center},
  decision/.style = {diamond, aspect=2.2, draw, align=center, inner xsep=1.2ex, inner ysep=1ex, text width=2cm},
  smldec/.style = {diamond, aspect=2.2, draw, align=center, inner xsep=1.2ex, inner ysep=1ex, text width=19mm},
  terminator/.style = {ellipse, draw, align=center, minimum height=8mm, minimum width=16mm},
  line/.style  = {->, line width=0.6pt}
}


%--- Title info
\title{SU update: \\ Two-point measurements at $Re_\tau \approx 5000$}
\author{JMO Massey$^{\dag}$, F Cabrera-Booman, T Jaroslawski, JC Klewicki, BJ McKeon}
\institute{Center for Turbulence Research \\ Stanford University}
% \thanks{This work was supported by DARPA under the CHAOS program}
\date{\today}

\begin{document}

%--- Title page
\begin{frame}
    \setcounter{framenumber}{0}
    \titlepage
    \vfill
    {\scriptsize \centering Thanks to DARPA for funding this work.\par}
\end{frame}

\begin{frame}
    \frametitle{Test matrix}
    Data completed thus far:
    \begin{table}[]
        \centering
        \begin{tabular}{lccccc}
        \toprule
        Pressure (psi) & 0 & 30 &  50 & 70&100 \\
        \midrule
        $Re_\tau$ & 1,500  &3,500& 5,000 &6,500& 8,000 \\
        \bottomrule
        \end{tabular}
    \end{table}

    % \begin{itemize}
    %     \item Fix $U_e=14[\mathrm{m/s}]$
    %     \item $\delta\approx 0.035[\mathrm{mm}]$
    % \end{itemize}
    Two point:
    \begin{table}[]
        \centering
        \begin{tabular}{lccccc}
        \toprule
        Pressure (psi) & 0  &  50 & 100 \\
        \midrule
        $Re_\tau$ & 1,500 & 5,000 & 8,000 \\
        \bottomrule
        \end{tabular}
    \end{table}

    \centering
    Callibration in-progress
\end{frame}


\begin{frame}
  \frametitle{Raw calibration signals}
  \begin{figure}
    \centering
    \includegraphics[width=0.6\textwidth]{sanity/50psi/PH-NKD/calib_ts_signals_50psi_nonoise.pdf}
    \includegraphics[width=0.6\textwidth]{sanity/50psi/PH-NKD/calib_spectra_50psi_nonoise.pdf}
    % \includegraphics[width=0.4\textwidth]{sanity/50psi/PH-NKD/calib_ts_signals_50psi_noise.pdf}
    % \includegraphics[width=0.4\textwidth]{sanity/50psi/PH-NKD/calib_ts_signals_50psi_noiseWN.pdf}
  \end{figure}
  % \centering
  % \textbf{Top left:} white noise, \textbf{top right:} \emph{only} facility noise, \textbf{bottom:} white noise + facility noise
\end{frame}

\begin{frame}
    \frametitle{TF and reconstructed spectra shows we need to filter}
        \centering
        \includegraphics[width=0.5\textwidth]{sanity/50psi/PH-NKD/H_50psi_nn.png}
        \includegraphics[width=0.5\textwidth]{sanity/50psi/PH-NKD/calib_spectra_50psi_nn_recon.pdf}
        % \includegraphics[width=0.4\textwidth]{sanity/50psi/PH-NKD/H_50psi_fn.png}
        % \includegraphics[width=0.4\textwidth]{sanity/50psi/PH-NKD/H_50psi_an.png}

        
\end{frame}

\begin{frame}
    \frametitle{After filtering the calibration signals match well}
        \centering
        \includegraphics[width=0.7\textwidth]{sanity/50psi/PH-NKD/calib_spectra_50psi_nn_filt_recon.pdf}
        % \includegraphics[width=0.4\textwidth]{sanity/50psi/PH-NKD/calib_spectra_50psi_fn_filt_recon.pdf}
        % \includegraphics[width=0.4\textwidth]{sanity/50psi/PH-NKD/calib_spectra_50psi_an_filt_recon.pdf}

        \centering
        In the ongoing measurements, we will apply this LP filter in real-time to avoid aliasing.
\end{frame}
% Check 

\begin{frame}
    \frametitle{2-point measurements at 50psi ($Re_\tau\approx 5000$)}
        \centering
        \includegraphics[width=0.4\textwidth]{sanity/50psi/03_10/pinholefig.pdf}

\end{frame}

\begin{frame}
    \frametitle{2-point measurements at 50psi ($Re_\tau\approx 5000$)}
        \centering
        \includegraphics[width=0.6\textwidth]{sanity/50psi/03_10/calib_ts_signals_50psi.pdf}
        \includegraphics[width=0.6\textwidth]{sanity/50psi/03_10/calib_ts_signals_50psi_part.pdf}

        \centering
        This demonstrates the need for a per-pinhole calibration (ongoing).
\end{frame}



\begin{frame}
    \frametitle{Raw spectra look promising}
        \centering
        \includegraphics[width=0.8\textwidth]{sanity/50psi/03_10/calib_spectra_50psi_f_filt_recon.pdf}
\end{frame}

\end{document}
