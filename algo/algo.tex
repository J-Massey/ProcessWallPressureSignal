
\documentclass[a4paper,11pt]{article}

\usepackage[british]{babel}
\usepackage[T1]{fontenc}
\usepackage[utf8]{inputenc}
\usepackage{geometry}
\usepackage{amsmath,amssymb}
\usepackage{siunitx}
\usepackage{graphicx}
\usepackage{booktabs}
\usepackage{hyperref}
\usepackage{algorithm}
\usepackage[noend]{algpseudocode}

\geometry{margin=1in}
\hypersetup{hidelinks}

\title{Wall- and Free-Stream Pressure Processing: Algorithmic Specification}
\author{Generated from the provided Python script}
\date{\today}

\algrenewcommand\algorithmicrequire{\textbf{Inputs:}}
\algrenewcommand\algorithmicensure{\textbf{Outputs:}}
\algrenewcommand\algorithmiccomment[1]{\hfill$\triangleright$ #1}

\begin{document}
\maketitle

\begin{abstract}
This document formalises, in pseudocode, the processing pipeline. The pipeline estimates a frequency response  transfer function ($H_1$) between a reference (free-stream) microphone and a treated (wall) microphone, then performs coherence-weighted Wiener inverse filtering to recover an estimate of the reference signal from the treated measurements.

The output amplitude from the speaker is still unknown at pressure so conversion from volts to decibels is TBD.
\end{abstract}

\section{Notation and Inputs}
    Let $x[n]$ denote the discrete-time \emph{reference} pressure signal, and $y[n]$ the \emph{treated}, both sampled at $f_s$~Hz. Frequency-domain quantities are indexed by $k$ for Welch frequency bins and by $r$ for the dense FFT grid used in deconvolution.

    \paragraph{Spectral definitions}
    Welch auto-spectra $S_{xx}[k]$, $S_{yy}[k]$ and cross-spectrum $S_{xy}[k]$ are computed with segment length $N_{\mathrm{seg}}$, overlap $N_{\mathrm{ov}}$, and Hann window $w[\cdot]$. The magnitude-squared coherence is
    \begin{equation}
        \gamma^2[k] \;=\; \frac{|S_{xy}[k]|^2}{S_{xx}[k]\,S_{yy}[k]}\in[0,1].
    \end{equation}
    The FRF estimate is $H[k] = S_{xy}[k]/S_{xx}[k]$.

\section{Algorithms}

\begin{algorithm}
    \caption{$H_1$ Transfer-Function Estimation}
    \label{alg:h1}
    \begin{algorithmic}[1]
        \Require Time series $x[n]$ (reference), $y[n]$ (treated), sampling rate $f_s$; Welch parameters $N_{\mathrm{seg}}, N_{\mathrm{ov}}, w[\cdot]$.
        \Ensure Frequency vector $f[k]$; complex FRF $H[k]$; coherence $\gamma^2[k]$.
        \State Optionally de-mean $x$ and $y$.
        \State Compute $S_{xx}[k]$ and $S_{yy}[k]$ using Welch's method with $(N_{\mathrm{seg}}, N_{\mathrm{ov}}, w)$.
        \State Compute cross-spectrum $S_{xy}[k]$ using the same Welch settings. \Comment{e.g.\ via a cross-spectral density routine}
        \State $H[k] \gets S_{xy}[k]/S_{xx}[k]$.
        \State $\gamma^2[k] \gets \frac{|S_{xy}[k]|^2}{S_{xx}[k]\,S_{yy}[k]}$, clipped to $[0,1]$.
        \State Return $(f[k], H[k], \gamma^2[k])$.
    \end{algorithmic}
\end{algorithm}

\begin{algorithm}
    \caption{Wiener Inverse Filtering for Trace Recovery}
        \label{alg:inv}
        \begin{algorithmic}[1]
        \Require Output $y[n]$; sampling rate $f_s$; FRF samples $(f[k], H[k])$; coherence $\gamma^2[k]$; optional band-limit $[f_{\min},f_{\max}]$; regulariser $\lambda\ge 0$; zero-padding $N_{\mathrm{pad}}\ge 0$.
        \Ensure Estimate $\hat{x}[n]$ of the input.
        \State De-mean $y$.
        \State $N \gets \text{length}(y)$; $N_{\mathrm{FFT}} \gets N + N_{\mathrm{pad}}$.
        \State $Y[r] \gets \mathrm{rFFT}\big(y, N_{\mathrm{FFT}}\big)$; $f_r \gets \mathrm{rFFTFreq}\big(N_{\mathrm{FFT}}, d=1/f_s\big)$.
        \State Interpolate $|H[k]|$ and $\phi[k]=\mathrm{unwrap}(\angle H[k])$ onto $f_r$ to obtain $|H|_r$ and $\phi_r$.
        \State $H_r \gets |H|_r\,e^{\mathrm{j}\phi_r}$.
        \State Interpolate $\gamma^2[k]$ onto $f_r$, clip to $[0,1]$ to get $\gamma^2_r$.
        \State Compute inverse filter (Wiener form)
        \begin{equation}
        H_r^{-1} \;\gets\; \gamma^2_r \,\frac{H_r^\ast}{|H|_r^2 + \lambda},
        \end{equation}
        or equivalently use $|H|_r^2 \leftarrow \max(|H|_r^2,\varepsilon)$ for numerical safety when $\lambda=0$.
        \If{$[f_{\min},f_{\max}]$ provided}
        \State Zero out $H_r^{-1}$ for all $f_r \notin [f_{\min},f_{\max}]$. \Comment{Band-limited inverse}
        \EndIf
        \State Set DC: $H_r^{-1}[0]\gets 0$.
        \State $\hat{x}[n] \gets \mathrm{iRFFT}\big(Y[r]\cdot H_r^{-1}, N_{\mathrm{FFT}}\big)[0{:}N]$.
        \State \Return $\hat{x}[n]$.
    \end{algorithmic}
\end{algorithm}

\begin{algorithm}
\caption{Wall/Free-Stream Pressure Processing Pipeline}
    \label{alg:pipeline}
    \begin{algorithmic}[1]
        \Require Index $i$ selecting a calibration file; processing constants $f_s,\nu_0,\rho_0,u_{\tau 0},\mathrm{W},\mathrm{He},L_0,\Delta L_0,U,C$; mode sets $\mathcal{M},\mathcal{N},\mathcal{L}$.
        \Ensure FRF plot and corrected-trace plot; arrays $(f,H,\gamma^2)$ and recovered trace $\hat{x}[n]$.
        \State Instantiate \texttt{WallPressureProcessor} with the given physical and processing parameters.
        \State Load calibration test $ (p_w[n], p_{fs}[n]) \leftarrow \text{load}(\texttt{fn\_naked\_pressures}[i])$.
        \State Set $\text{ref}\gets p_w,\;\text{trt}\gets p_{fs}$.
        \State $(f,H,\gamma^2) \gets \textsc{H1 Transfer-Function Estimation}(\text{ref}, \text{trt}, f_s)$ \Comment{Alg.~\ref{alg:h1}}
        \State Save a transfer-function figure: magnitude and phase of $H$ vs.\ $f$, optionally with coherence overlay.
        \State $\hat{x}[n] \gets \textsc{Wiener Inverse Filtering}(\text{trt}, f_s, f, H, \gamma^2, [f_{\min}, f_{\max}], \lambda, N_{\mathrm{pad}})$ \Comment{Alg.~\ref{alg:inv}}
        \State Define $t[n] \gets n/f_s$.
        \State Save a corrected-trace figure showing $ \{\text{ref}(t), \text{trt}(t), \hat{x}(t)\}$.
    \end{algorithmic}
\end{algorithm}

\section{Default Parameters (from the Script)}
Unless otherwise stated, the following defaults are used in the reference implementation:
\begin{center}
    \begin{tabular}{@{}ll@{}}
        \toprule
        Quantity & Value \\ \midrule
        Sampling rate $f_s$ & \SI{25000}{Hz} \\
        % Kinematic viscosity $\nu_0$ & $1.52\times 10^{-5}\,\mathrm{m^2\,s^{-1}}$ \\
        % Density $\rho_0$ & \SI{1.225}{kg\,m^{-3}} \\
        % Friction velocity $u_{\tau 0}$ & \SI{0.358}{m\,s^{-1}} \\
        % Uncertainty fraction & $\pm 3\%$ \\
        % Duct width $W$ & \SI{0.30}{m} \\
        % Duct height $H_e$ & \SI{0.152}{m} \\
        % Duct length $L_0$ & \SI{3.0}{m} \\
        % Low-frequency end correction $\Delta L_0$ & $0.1\,L_0$ \\
        % Flow speed $U$ & \SI{14.2}{m\,s^{-1}} \\
        % Speed of sound $C$ & $\sqrt{1.4\cdot 101325/\rho_0}$ \\
        % Mode sets $(m,n,\ell)$ & $[0], [0], [0,1,4,5,8,11,15]$ \\
        Welch defaults & $N_{\mathrm{seg}}{=}4096,\; N_{\mathrm{ov}}{=}2048,\; w{=}$ Hann \\
        Inverse filter band (example) & $(0, 3000]\,\mathrm{Hz}$ \emph{optional} \\ \bottomrule
    \end{tabular}
\end{center}

% \section{Complexity and Numerical Notes}
% \begin{itemize}
%   \item Welch spectral estimates dominate runtime: $O(N\log N)$ per segment; inverse filtering adds an FFT/iFFT pair of size $N_{\mathrm{FFT}}$.
%   \item Unwrapping the FRF phase before interpolation avoids $2\pi$ discontinuities and artefacts in the inverse.
%   \item Coherence weighting $\gamma^2$ attenuates frequency bins where the input--output relation is weak, improving robustness to noise.
%   \item Regularisation: choose $\lambda>0$ to stabilise the inverse near FRF minima; setting $\lambda=0$ with a small floor $\varepsilon$ is equivalent to the provided script.
%   \item Zero the DC gain of the inverse to avoid ill-conditioning after de-meaning.
% \end{itemize}

\end{document}